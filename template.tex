%   DOCUMENT CLASS  %%%%%%%%%%%%%%%%%%%%%%%%%%%%%%%%%%%%%%%%%%%%%%%%%%%%%%%%%%%
%
%   Use the `sfuthesis` class to format your thesis. If your program does not
%   require a thesis defence, use the class option `undefended` like so:
%
%     \documentclass[undefended]{sfuthesis}
%
%   To generate a signature page for your defence, use the `sfuapproval` class
%   instead, by replacing the below line with
%
%     \documentclass{sfuapproval}
%
%   For more information about thesis formatting requirements, go to
%
%     http://www.lib.sfu.ca/help/publish/thesis
%
%   or ask a thesis advisor at the SFU Research Commons.
%

\documentclass{sfuthesis}



%   DOCUMENT METADATA  %%%%%%%%%%%%%%%%%%%%%%%%%%%%%%%%%%%%%%%%%%%%%%%%%%%%%%%%
%
%   Fill in the following information for the title page and approval page.
%

\title{A Report on the Intersection of Long-Term Autonomy and Human-Robot Interaction Research}
\thesistype{Depth Report}
\author{Jack Thomas}
\previousdegrees{%
	M.Math, University of Waterloo, 2014\\
	B.A., University of New Brunswick, 2012\\
	B.CS., University of New Brunswick, 2012}
\degree{Doctor of Philosophy}
\discipline{Computer Science}
\department{Department of Computing Science}
\faculty{Faculty of Applied Science}
\copyrightyear{2018}
\semester{Spring 2018}
\date{TBD, 2018}

\keywords{robotics; human-robot interaction; long-term autonomy}

\committee{%
%	\chair{Pamela Isely}{Professor}
%	\member{Emmett Brown}{Senior Supervisor\\Professor}
%	\member{Bonnibel Bubblegum}{Supervisor\\Associate P%rofessor}
%	\member{James Moriarty}{Supervisor\\Adjunct Professor}
%	\member{Kaylee Frye}{Internal Examiner\\Assistant Professor\\School of Engineering Science}
%	\member{Hubert J.\ Farnsworth}{External Examiner\\Professor\\Department of Quantum Fields\\Mars University}
}



%   PACKAGES %%%%%%%%%%%%%%%%%%%%%%%%%%%%%%%%%%%%%%%%%%%%%%%%%%%%%%%%%%%%%%%%%%
%
%   Add any packages you need for your thesis here.
%   You don't need to call the following packages, which are already called in
%   the sfuthesis class file:
%
%   - appendix
%   - etoolbox
%   - fontenc
%   - geometry
%   - lmodern
%   - nowidow
%   - setspace
%   - tocloft
%
%   If you call one of the above packages (or one of their dependencies) with
%   options, you may get a "Option clash" LaTeX error. If you get this error,
%   you can fix it by removing your copy of \usepackage and passing the options
%   you need by adding
%
%       \PassOptionsToPackage{<options>}{<package>}
%
%   before \documentclass{sfuthesis}.
%
%   The following packages are a few suggestions you might find useful.
%
%   (1) amsmath and amssymb are essential if you have math in your thesis;
%       they provide useful commands like ``blackboard bold'' symbols and
%       environments for aligning equations.
%   (2) amsthm includes allows you to easily change the style and numbering of
%       theorems. It also provides an environment for proofs.
%   (3) graphicx allows you to add images with \includegraphics{filename}.
%   (4) hyperref turns your citations and cross-references into clickable
%       links, and adds metadata to the compiled PDF.
%   (5) pdfpages lets you import pages of external PDFs using the command
%       \includepdf{filename}. You will need to do this if your research
%       requires an Ethics Statement.
%

\usepackage{amsmath}                            % (1)
\usepackage{amssymb}                            % (1)
\usepackage{amsthm}                             % (2)
\usepackage{graphicx}                           % (3)
\usepackage[pdfborder={0 0 0}]{hyperref}        % (4)
% \usepackage{pdfpages}                         % (5)
% ...
% ...
% ...
% ... add your own packages here!




%   OTHER CUSTOMIZATIONS %%%%%%%%%%%%%%%%%%%%%%%%%%%%%%%%%%%%%%%%%%%%%%%%%%%%%%
%
%   Add any packages you need for your thesis here. We've started you off with
%   a few suggestions.
%
%   (1) Use a single word space between sentences. If you disable this, you
%       will have to manually control spacing around abbreviations.
%   (2) Correct the capitalization of "Chapter" and "Section" if you use the
%       \autoref macro from the `hyperref` package.
%   (3) The LaTeX thesis template defaults to one-and-a-half line spacing. If
%       your supervisor prefers double-spacing, you can redefine the
%       \defaultspacing command.
%

\frenchspacing                                    % (1)
\renewcommand*{\chapterautorefname}{Chapter}      % (2)
\renewcommand*{\sectionautorefname}{Section}      % (2)
\renewcommand*{\subsectionautorefname}{Section}   % (2)
% \renewcommand{\defaultspacing}{\doublespacing}  % (3)
% ...
% ...
% ...
% ... add your own customizations here!




%   FRONTMATTER  %%%%%%%%%%%%%%%%%%%%%%%%%%%%%%%%%%%%%%%%%%%%%%%%%%%%%%%%%%%%%%
%
%   Title page, committee page, copyright declaration, abstract,
%   dedication, acknowledgements, table of contents, etc.
%
%   If your research requires an Ethics Statement, download one from the
%   SFU library website and uncomment the appropriate lines below.
%

\begin{document}

\frontmatter
\maketitle{}
\makecommittee{}

%\addtoToC{Ethics Statement}%
%\includepdf[pagecommand={\thispagestyle{plain}}]{ethicsstatement.pdf}%
%\clearpage

\begin{abstract}
The abstract should introduce that human-robot interaction and long-term autonomy have an area of overlap of growing importance, as robots both become more self-sufficient and push deeper into human environments like workplaces and public spaces. It should preview that we will look for relevant work from both sides, particularly any that addresses issues specific to this intersection, without jumping to any of our conclusions about the problem as a whole.
\end{abstract}


%\begin{dedication}
%	This is an optional page.
%\end{dedication}


%\begin{acknowledgements}
%	This is an optional page.
%\end{acknowledgements}

\addtoToC{Table of Contents}%
\tableofcontents%
\clearpage

\addtoToC{List of Tables}%
\listoftables%
\clearpage

\addtoToC{List of Figures}%
\listoffigures%
\clearpage





%   MAIN MATTER  %%%%%%%%%%%%%%%%%%%%%%%%%%%%%%%%%%%%%%%%%%%%%%%%%%%%%%%%%%%%%%
%
%   Start writing your thesis --- or start \include ing chapters --- here.
%

\mainmatter%

\chapter{Introduction}

The introduction will expand on the abstract by going into more detail about the definitions of human-robot interaction and autonomy/LTA, as well as discussing some of the specific examples of robots and situations where their intersection is relevant (autonomous taxis, warehouse robots, domestic robots, getting onto public transit or even through a door). It should be short, saving the heavy lifting for the next section and may involve no or few references of its own, except possibly to provide some key definitions or a few touchstone authors/papers to help the reader get their bearing.

Three Engineers, Hundreds of Robots, One Warehouse (Kiva 2008 example)
https://spectrum.ieee.org/robotics/robotics-software/three-engineers-hundreds-of-robots-one-warehouse

The cocktail party robot: Sound source separation and localisation with an active binaural head

The strands project: Long-term autonomy in everyday environments

\section{Background}

After the intro's high-level overview, this section of the Introduction goes into more detail scoping and justifying why we think the intersection area of HRI and LTA is a cohesive topic in the literature presenting unique and interesting problems.

-The STRANDS project (likely to reappear later in LTA)
-Possibly some supporting references on robots entering the workplace, like Amazon's warehouses, autonomous 
-Whoever originally said robots were for "dull, dirty or dangerous" work
-braitenberg 's Vehicles (maybe a cheeky reference but I'm sure I can find some relevance about the blurry line between a simple machine and an autonomous agent)
-How to make robots that make friends and influence people (that particular argument about cultivating in humans the belief that the robot has intentions)
-Autonomous Agents: From Biological Inspiration to Implementation and Control (for a fairly uncontroversial definition of robot autonomy)
-Pfeifer, Building Fungus Eaters
-Tim Smithers

Building "Fungus Eaters": Design Principles of Autonomous Agents	Rolf Pfeifer's work about what it takes to be autonomous. Good starting point to build toward what it takes to be autonomous among people.

Elephants Don't Play Chess	Another important Rodney Brooks paper, part of the philosophy of embodied, modern AI over classical.

Vehicles	Brietenberg, almost exclusively about our projections onto a machine of lifelike characteristics

How to build robots that make friends and influence people	Immediately makes point that robots need to convey intentionality, that humans recognize them. Important to independent autonomy

The Principal-Agent Problem

Collective Action Problems

Ants don't have Friends - Thoughts on Socially Intelligent Agents

Autonomy in Robots and Other Agents (TIM SMITHERS)

AN EXISTING, ECOLOGICALLY-SUCCESSFUL GENUS OF COLLECTIVELY INTELLIGENT ARTIFICIAL CREATURES

Mixing human and non-humans togetehr: The sociology of a door closer

Android arete: Toward a virtue ethic for computational agents

When are robots intelligent autonomous agents?

\chapter{Autonomy}

Having scoped and grounded our area of interest in the previous chapter, this chapter gets down to brass tacks on research done to expand robots' autonomy.

A fast and frugal method for team-task allocation in a multi-robot transportation system

Speaking Swarmish: Human-Robot Interface Design for Large Swarms of Autonomous Mobile Robots

Designing interfaces for multi-user, multi-robot systems	Design principle study for mixed human-robot teams, useful when discussing autonomy	

Probabilistic robotics?

(More Swarm, Multi-robot system, autonomous robotics research in high level and general)

\section{Long-Term Autonomy}

This section should demonstrate some of the breadth of issues in LTA research, such as scheduling, power management, dealing with failures, etc. The end of this section should highlight the importance of navigation to almost every project.

Robots in the Wild: Understanding long-term use	Longitudinal study of 30 households with Roombas, answering question of whether people keep using them when novelty wears off. Novelty wearing off is a big concern for long-term autonomy	

Long-Term Autonomy in Office Environments	A solid case-study of running a robot in a lab for two weeks, letting it recharge itself and seeing when it would need and request help

How Robot Products become Social Products: An Ethnographic study of cleaning in the home.	Study of how families adapted to the Roomba, possibly useful in discussing how people adapt to the presence of robots in their lives.	

Service Robots in the Domestic Environment: A Study of the Roomba Vacuum in the Home	It's about the Roomba! It's obviously relevant! Ethnography on how users long-term get along with service robots in the home.	

Life-Long Learning of daily Human Routine in Home Environments	Applying learning and adaptation to human behaviour, a good example of a long-term project adapting itself.

The Snackbot: Documenting the design of a robot for long-term human-robot interaction	LONG-TERM HRI! Not multi-robot and maybe a little bare-bones, but here's a good example!

Autonomous door opening and plugging in with a personal robot	Combines door navigation with recharging task, which ties the activities together into LTA

Behaviors for Robust Door Opening and Doorway Traversal with a Force-Sensing Mobile Manipulator		Pure doorway navigation

Staying Alive Longer: Autonomous Robot Charging Put To The Test

The mobot museum robot installations: a five year experiment

Experiences with an Autonomous Robot Attending AAAI

Autonomous door opening and plugging in with a personal robot

Basic cycles, utility and opportunism in self-sufficient robots

Competitive Foraging, Decision Making, and the Ecological Rationality of the Matching Law

Lessons learned from robotic vacuum cleaners entering the home ecosystem

\section{Navigation}

This is not the only section in which robot navigation is discussed, but should be the one explaining the core concept apart from when humans are involved. The references here should explain how autonomous navigation is a fundamental, perhaps even defining part of being an autonomous robot, and has therefore been a central focus of robotics research for decades. Papers that focus on navigation by lone robots or by fully-controlled multi-robot systems are appropriate, as are ones about navigation algorithms, ideally examples that do not consider other robots or humans or else treat them as dynamic obstacles instead of interacting agents.

The terminal references of this section will be a discussion of Reciprocal Velocity Obstacles and some of the research related to that, as well as crowd modelling from the likes of Dinesh Manocha. These provide a natural segue to the start of the next chapter by highlighting work that addresses the reciprocal and cooperative elements of navigation.

-Shaky
-Dynamic Window Algorithm
-Reciprocal Velocity Obstacles
-Dinesh's Crowd Modelling

Shakey the Robot	It's Shakey! Just a good part of robotic history, important to reference while giving background

The Dynamic Window Approach to Collision Avoidance	That classic paper from the directed reading, influential on robot navigation for problems like getting through a door.	

Reciprocal velocity obstacles for real-time multi-agent navigation

Bravo: Biased reciprocal velocity obstacles break symmetry in dense robot populations

Reducing spatial interference in robot teams by local-investment aggression

\chapter{Human-Robot Interaction}

This chapter will explore human-robot interaction starting from the lead given at the end of the last chapter on navigation. The goal is not to survey all of human-robot interaction but to follow a thread outward from navigation toward what features make for a more successful social human interaction.

To bridge from one chapter to the other, I'll include references for navigation involving biological inspiration (ants, herding, flocking, aggression, all sounds pretty familiar), many of which may use biology as a guide but do not actually use the behaviour to interact with the original species. These references should set up the idea that human-inspired does not necessarily equal human-compliant.

Timing in Multimodal turn-taking Interactions: Control and Analysis using Timed Petri Nets	HRI case dealing with turn-taking interaction styles.

Reasoning for a multi-modal service robot considering uncertainty in human-robot interactions	POMDPs, remember them from 

Joelle? A reasoning system to fill the gap between sensors/actuators and POMDPs, multi-modal fusion

Survey of metrics for human-robot interaction	Thank goodness, a survey of dozens of methods to measure human-robot interaction.

Using Motives and Artificial Emotion for Long-Term Activity of an Autonomous Robot


\section{Human-Compliant Navigation}

Compliance literature seems relatively recent and deals in proxemics This is also an opportunity to discuss the machine-learning angle of many of these systems, which are trained on either simulation or real-world data. This will help us later create a juxtaposition between continuous-space, implicit, learned social interactions and discrete, explicit, designed social behaviours.

-What was that one collision avoidance behaviour paper from IROS 2017?
-Stanford's Jackrabbot (several papers)
-Burgard's social compliance reinforcement learning

Human-Aware Navigation: A Survey	Similar to that social robotics paper, a survey and classification of this type of project

Evaluation of Passing Distance for Social Robots	Another proxemic user study, interesting finding that too much lateral distance is also weird besides too little	


Towards more efficient navigation for robots and humans	Almost exactly what we're talking about, a user study, social cues to signal, passing, although through corridors	

Making a case for spatial prompting in Human-Robot Communication	Talks about how people configure their relative position to the robot based on situation, and how to nonverbally prompt them to take up a certain position	

Socially compliant mobile robot navigation via inverse reinforcement learning

A Predictive Collision Avoidance Model for Pedestrian Simulation

Towards a socially acceptable collision avoidance for a mobile robot navigating among pedestrians using a pedestrian model

Toward Understanding Social Cues and Signals in Human-Robot Interaction: Effects of Robot Gaze and Proxemic Behaviour	Corridor passing, proxemic was influential while gaze was not, multiple interactions were more effective	

 Human-friendly robot navigation in dynamic environments 

Local reactive robot navigation: A comparison between reciprocal velocity obstacle variants and human-like behavior 

Viewing Robot Navigation in Human Environments as a Cooperative behaviour

Socially Aware Motion Planning with Deep Reinforcement Learning

Dynamic path planning adopting human navigation strategies for a domestic mobile robot

Natural person-following behaviour for social robots	Looking at how people handle having a robot following them around.	

Human-Robot Proxemics: Physical and psychological distancing in human-robot interaction	Adapting social norms to maintaining distance between social interlocuters.

Walking Together: Side by Side Walking Model for an Interacting Robot	Study and theory about walking next to someone that goes beyond "maintain distance". Predicts, obstacle avoids.

Directly or on detours?: how should industrial robots approximate humans?	Study where workplace robots either went straight to work or wandered around. Surprise, people don't like the wandering around!	

Socially Adaptive Path Planning in Human Environments Using Inverse Reinforcement Learning	From the NCFRN field trials, great example of developing human-conscious technique rather than max-efficiency path.	

Socially-Driven Collective Path-Planning for Robot Missions	Concept of fairness used to decide which waypoints to visit when multiple humans have each designated stops but there's too many

Social Forces Model for Pedestrian Dynamics

Robot Navigation in Dense Human Crowds: the Case for Cooperation

Friendly Patrolling: A Model of Natural Encounters	A study on how to make robot behaviour approacheable and inviting, useful in the workplace	2011

Understanding suitable locations for waiting	Long-running systems will have to handle staying out of the way quite often, so an example system that looks for places to wait is smart.	

Designing interruptive behaviours of a public environmental monitoring robot	Interesting
 case, where the robot needs to be assertive in a way that will get humans to listen to its instructions.	

Autonomous Pedestrian-like navigation in city centers	A good HRI bend to this particular talk, since long-term autonomy means blending in with crowds.	

How to Approach Humans? Strategies for Social Robots to Initiate Interactions	Describes the challenges of getting a walking person's attention as a robot and details how they achieved it. Useful for long-term systems.	

Footing in Human-Robot Conversations: How Robots Might Shape Participant Roles using Gaze Cues	Study evidence that a robot, if perceived as an autonomous, distinct agent via the use of human cues, can assert social roles that people will conform to.	

Perception of Affect Elicited by Robot Motion	Suggests acceleration and curvature are enough to infer emotional affect, even in non-humanoid robots. Non-verbal communication possible with non-human robots	

An Analysis of Deceptive Robot Motion	The reasons for and means of generating deceptive robot motion. Useful in study of disagreeing robots	2014

Exploring Interruption in HRI using Wizard of Oz	Not sure of significance of work, but topic of exploring when to interrupt humans as a robot is relevant.	

Nonverbal leakage in robots: communication of intentions through seemingly unintentional behaviour	Useful to establish that not all of our communication is deliberate or conscious, and robots need to participate in these nonverbal cues to be understood	

Nonverbal robot-group interaction using an imitated gaze cut	Signalling a delivery to an unsuspecting recipient


\section{Social HRI}

This section widens the scope of our HRI interest beyond navigation to any paper where the robot is engaged in a social interaction with a human. This can still be a wide area, so apart from some major papers and a survey or two, papers that focus on explicit one-on-one interactions are to be favored. The section should hopefully suggest that human acceptance is an important factor to successful social interactions, and raise questions about what characteristics promote cooperation from users.

Spotting social interaction by using the robot energy consumption

Trust-Driven Interactive Visual Navigation for Autonomous Robots	Early trust-work, modelling robot's estimation of current trust and modifying autonomy level in response.

OPTIMo: Online Probabilistic Trust Inference Model for Asymmetric Human-Robot Collaborations	An ongoing trust model. 

Personalizes to different users, Bayesian techniques employed. "Near-real-time"?

Exploring socially intelligent recharge behaviour for human-robot interaction	Since recharging is something robots are going to have to do, studying how to manage people's social reaction to it is useful.

Managing social constraints on recharge behaviour for robot companions using memory	The learning technique to figure out when's a good time to recharge isn't that interesting, but the idea of picking the right time to recharge is.


Social Robots for Long-Term Interaction: A Survey Basically a mandatory survey to reference the breadth of the research area the depth report covers (at least, from 2013) Note: Two more similarly named HRI surveys, but from 2003 and 2007, so pretty out of date

Personalization in HRI: A Longitudinal Field Experiment	Long-term study on the relevance of building a rapport with humans and personalizing the robot's interactions to match their 'relationship'. Autonomy relevance!

Social vs. Useful HRI: Experiencing the Familiar, Perceiving the Robot as a Sociable Partner and Responding to Its Actions

Social interactions in HRI: the robot view (Breazeal)

Socially Aware Motion Planning with Deep Reinforcement Learning 

A social robot that stands in line

Robots asking for directions - The Willingnesss of passers-by to support robots	A useful study to make the point about autonomy, and the importance of getting people to recognize robots if you want them to cooperate

Socially Distributed Perception	Integrating asking for help into a robot's problem-solving process

Toward sociable robots

Designing robots for long-term social interaction

Kismet

?From Hal to Kismet: Your Evolution Dollars at Work? (Critic of Kismet,Breazeal and Brooks)

Human-Robot Teams Collaborating Socially, Organizationally and Culturally	Interesting idea about three levels of customs for autonomous robots to identify and modify behaviour around.	

Sorry Dave, I'm Afraid I Can't Do That: Explaining Unachievable Robot Tasks Using Natural Language	Example of robots having to disagree/decline user instructions as impossible using natural language	2013

\section{Human Factors}

This section widens the discussion again to human factor research on robot acceptance and perception by humans, looking at issues like trust, novelty, anthropomorphism, cultural differences and anything else that might explain why people may or may not choose to cooperate with a robot.

A subsection on HCI and interface design is warranted to discuss relevant research outside of robotics. This provides a way to include the audio references and any other literature on workplaces that don't specifically address robots. 


-Perhaps the one about how to handle receiving too many navigation waypoints to visit by too many human users, as an example of a robot making their own decisions about which humans to obey (and implicitly that humans would have to accept their judgement on the matter)
-The study about the robot guiding a crowd to a fire escape when a seemingly more direct route is available.
-Anqi Xu's work on trust (particularly recognizing trust being lost and trying to signal that they've learned their lesson and would like trust back)
-The earcon paper about the cola factory split between two people

Taking candy from a robot: Speed features and candy accessibility predict human response


"Daisy, Daisy, give me your answer do!" Switching off a robot	People recognize "Animancy" of robot, might be relevant to philosophical argument about recognition of robot being tied to autonomy	

Decision-Making Authority, Team Efficiency and Human Worker Satisfaction in Mixed Human-Robot Teams	Interesting work where humans are willing to give up scheduling autonomy if robot does it more efficiently	2014

Effects of framing a robot as a social agent or as a machine on children's social behavior

Evidence that Robots Trigger a Cheating Detector in Humans	They really narrow down on cheating to win (as opposed to other cheating or just the motion involved in cheating) makes an agent seem more autonomous.	

Critic, compatriot or chump? Response to robot blame attribution	Not a surprising result, NOBODY likes getting blamed, but notable since autonomy means being able to stand up to yourself in human interactions	

Towards Industrial Robots with Human-like Moral Responsibilities	Proposing moral safeguards in the workplace that go beyond safety procedures. Concept of a robot that can be responsible is relevant to autonomy.	

Robots that express emotion elicit better human teaching	Useful to prove the value of social interaction and cultivating emotional bonds with humans	

Do people hold a humanoid robot morally accountable for the harm it causes?	Robot held more accountable than vending 

machine, less than human. So there are grades of autonomy we recognize (sheepdog example)	

When the robot criticizes you…: Self-serving bias in human-robot interaction	Another reminder that if you don't respect 

the robot's autonomy you won't respect negative feedback	

Effects of speech on perceived capability	Just a little study that shows robots that are conversational and responsive seem more capable than ones that don't communicate. Even when the communication is irrelevant, just being engaged increases people's perception of the robot.	

On the Effect of the user's background on communicating grasping commands	Difference between technical and non-technical users and how they communicate with robots.

Beyond Dirty, Dangerous and Dull: What Everyday People Think Robots Should Do	Good for backing up some assertion about peoples' expectations of robots in the workplace	


Eyewitnesses are misled by human but not robot interviewers	Large number of respondents were shown a video then questioned by either a human or a robot who tried to mislead them about it. Robot did worse.	

Rabble of Robots Effects: Number and Type of Robots Modulates Attitudes, Emotions, and Stereotypes	Might be more useful as a psychological study example, but they do have a point that groups of robots won't be treated the same as individuals. Did they explore heterogeneous groups?	

The Advisor Robot: Tracing People's Mental Model From a Robot's Physical Attributes	A study with four different robot head and different voices to examine impact on humans, design principles, "uncanny valley" possibly?	

Affective Expression in Appearance-constrained Robots	2-pager, non-anthropomorphic robots can still express themselves

Judging a Bot By Its Cover: An Experiment on Expectation Setting for Personal Robots	A simple reference for how end users don't have the same appreciation for a robot's capabilities as we might, and how important it is to manage expectations	

A peer pressure experiment: Recreation of the Asch conformity experiment with robots

When in Rome: the role of culture & context in adherence to robot recommendations

Which Robot Am I Thinking About?: The Impact of Action and Appearance on People's Evaluations of a Moral Robot	

The influence of people’s culture and prior experiences with Aibo on their attitude towards robots

Backchannel opportunity prediction for social robot listeners

Cultural differences in attitudes towards robots

Three's company, or a crowd?: The effects of robot number and behavior on HRI in Japan and the USA	Underlining different reactions to groups vs. individuals and social vs. functional robots	2015

How a robot should give advice	Advice given as friendly suggestion more effective than given as orders. Case for robotic politeness, courtesy	

\section{Interfaces}

Earcons and icons: Their structure and common design principles

Comparing Heads-up, Heands-Free Operation of ground robots to teleoperation

A detailed investigation into the effectiveness of earcons

The effects of interruptions on task performance, annoyance, and anxiety in the user interface

Using Vision, Acoustic and Natural language for Disambiguation	Multimodal! Use multiple input streams to resolve ambiguities in one stream.

Chorusing and Controlled Clustering for Minimal Mobile Agents	Using audio chirps to regulate a swarm's cluster size.	


Listening to vs overhearing robots in a hotel public space	OVERHEARING! Two robots talking to each other who say things people might want to know!	

Effective Sounds in Complex Systems: The Arkola Situation   A signature HCI audio paper from the Earcon line, used in the audio project, good example of an hci paper with findings useful to HRI.

The Roomba mood ring: An ambient display robot	An oddity, people vote on what colour the robot displays as a way to gauge the current emotional tenor of the room.

\chapter{Conclusion}

The conclusion should go beyond a dry summary of the areas the depth report has covered. While the second chapter worked to justify our interest in the intersection of the following two chapters, it will be necessary to revisit that justification at the end now that the reader has hopefully absorbed enough material to be persuaded of it. By leveraging what human interaction research has learned about the requirements for gaining human, we can again argue that the goals of long-term autonomy in human environments (particularly concerning navigation) will require explicit, intentional design of recognition- and reciprocation-inducing behaviours.

It is also partly the purpose of a depth report to point the way toward the eventual thesis. In this case, the depth report will hopefully have suggested that there is a rich vein to be mined figuring out how to promote recognition, reciprocation, and ultimately acceptance from human users for robots' social agency in order to enable their navigational autonomy as part of achieving Long Term Autonomy.



%   BACK MATTER  %%%%%%%%%%%%%%%%%%%%%%%%%%%%%%%%%%%%%%%%%%%%%%%%%%%%%%%%%%%%%%
%
%   References and appendices. Appendices come after the bibliography and
%   should be in the order that they are referred to in the text.
%
%   If you include figures, etc. in an appendix, be sure to use
%
%       \caption[]{...}
%
%   to make sure they are not listed in the List of Figures.
%

\backmatter%
	\addtoToC{Bibliography}
	\bibliographystyle{plain}
	\bibliography{references}

\begin{appendices} % optional
	\chapter{Code}
\end{appendices}
\end{document}
